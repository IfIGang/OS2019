\section{Checkpoint 2}

% align with slide numbers
\addtocounter{subsection}{1}

\subsection{What is Preemption?}
In einem System in dem Prozesse in Zeitscheiben unterteilt werden, ist dies die faire Aufteilung dieser Zeitscheiben auf die Prozesse, durch das Unterbrechen des aktuellen Prozesses vom Betriebssystem.
Jeder Prozess erhält ungefähr gleich viel Rechenzeit.

\subsection{What new challanges did Preemption introduce when compared to cooperative multiprogramming?}
Programme können an beliebigen Stellen unterbrochen werden.
Dies wird zu einem Problem, falls eine Resource zwischen verschiedenen Programmen geteielt wird und beide Programme darauf zugreifen.

\subsection{How is Preemption implemented in an operating system kernel?}
Meist durch einen Timer Interrupt. (Eine Zeitscheibe endet, wenn die Clock ein Signal gibt.)
Der Scheduler wählt ein neues Programm aus, welches Rechenzeit erhalten soll und der Dispatcher tauscht den aktiven Kontext aus, damit das neue Programm rechnen kann.

\subsection{Compare Concurrency and Parallelism.}
Concurrency: keine echte Gleichzeitigkeit, sondern eher ein konstantes abwechseln (immer nur aktiver Thread)\\
Parallelism: zwei Kerne, die echt Gleichzeitig rechnen

\subsection{What is a Critical Section?}
Ein Codesegment, in dem auf eine geteilte Resource zugegriffen wird.

\subsection{What is the value of shared, and why?}
In einer Nebenläufigen Ausführung von func\_a und func\_b:

\lstset{language=C}
\begin{minipage}{.5\textwidth}
\begin{lstlisting}
int shared = 0;
void func_a(void)
{
    shared++;
}
\end{lstlisting}
\end{minipage}
\begin{minipage}{.5\textwidth}
\begin{lstlisting}

void func_b(void)
{
    shared--;
}
\end{lstlisting}
\end{minipage}

Antwort: -1, 0 oder 1; je nachdem wie die Unterbrechung der Programme erfolgen, da das keine atomaren Anweisungen sind und wir uns somit temporäre Werte in Registern merken müssen, welche durch die Unterbrechung invalide werden.

\subsection{Describe the three criteria that correct solutions of the Critical Section problem must fulfill.}
\begin{itemize}
    \item Progress: wenn mehrere Prozesse in die Critical Section eintreten wollen, dann darf die Entscheidung wer eintreten darf nicht beliebig lange hinausgezögert werden.
    \item gegenseitiger Ausschluss: wenn ein Prozess in der Critical Section ist, darf kein anderer Prozess in seiner Critical Section sein.
    \item Bounded Waiting: wenn ein Prozess in die Critical Section eintreten möchte, dann darf er nicht unendlich lange daran gehindert werden. (aber beliebig lange)
\end{itemize}

\subsection{Why does this naive approach not solve the critical section problem? Outline a schedule.}
\begin{lstlisting}
shared int turn = 0;
do { // Code for T_i
    while (turn != i);
        critical section
    turn = j;
        remainder section
} while (1);
\end{lstlisting}
Problem: Progress wird verletzt: wenn ein Thread einen anderen ``überholt'', dann kann es passieren, dass dieser Thread am Eintreten der Critical Section gehindert wird, da der andere Thread noch in der remainder Section ist.
Der erste Thread muss warten, bis der zweite seine Remainder Section und Critical Section durchlaufen hat, bis er selbst in die Critical Section eintreten darf.

Ein Beispiel: zwei Prozesse P0 und P1.
turn ist auf 0 und P0 ist in der critical section und P1 in der remainder section.
P0 verlässt nun seine critical section und turn wird dadurch auf 1 gesetzt.
Wenn P0 nun vor P1 die remainder section verlässt, kann P0 nicht in die critical section eintreten, da turn immer noch auf 1 ist.
P0 wird somit am Eintreten gehindert und Progress wird verletzt.

\subsection{Name and describe one well known software algorithm that solves the critical section problem.}
verschiedene Möglichkeiten:

Baker Algorithmus:
Jeder Prozess zieht eine Nummer, wenn er in die critical Section eintreten will.
Die gezogenen Nummer sind monoton steigend, aber nicht streng-monoton steigend.
Es darf immer der Prozess eintreten, der die kleinste Nummer hat, oder bei Gleichstand zwischen zwei Prozessen der mit der kleinsten ID\@.

Peterson Algorithmus

Dekers Algorithmus
\todo genauere Erklärungen

\subsection{Discuss whether pure software algorithms are a good solutions to the critical section problem.}
Nein, denn sie funktionieren nicht für echte Parallelität (Parallelism) und sie sind langsam mit vielen Threads.

\subsection{To mitigate the problems of the software algorithms, hardware approaches are used instead. Which is not one of them?}
\begin{itemize}
    \item[a:] test-and-set
    \item[b:] outlook
    \item[c:] exchange (compare and swap)
    \item[d:] interrupt disabling   
\end{itemize}
Antwort:
d wird in Windows genutzt um die Critical Section vor Unterbrechungen abzusichern.
a und c nutzen beide das Prinzip eine Variable atomar zu setzen und nur in die Critical Section einzutreten, wenn die Operation ein Erfolg war.
Entsprechend ist c falsch.

\subsection{What is a Semaphore?}
Eine geteilte \underline{Integer} Variable, deren Zugriff durch das Betriebssystem geschützt wird.
(Ihre Operationen sind atomar.)

\subsubsection{What operations are defined on Semaphores?}
wait (Dekrementieren des Wertes, wenn die Semaphore > 0 ist), signal (Inkrementieren)

\subsection{Identify the advantages and disadvantages of using Semaphores over native hardware approaches.}
Vorteil: Semaphore ist Abstraktion, damit ist die zugrundeliegende Hardware / Betriebssystem irrelevant; die Semaphore kann mehrere Zustände haben; weniger Bussy Waiting als bei test-and-set\\
Nachteil: (Code) Overhead gegenüber test-and-set

\subsection{What is a Deadlock?}
Mehrere Prozesse warten auf eine Freigabe (von z.B. einer Semaphore), die durch einen jeweils anderen wartenden Prozess bereits blockiert ist.
Entsprechend kann kein Prozess weiter arbeiten. 

\subsubsection{Describe how a Deadlock can be the result of careless use of Semaphores.}
Beispiel: Zwei Prozesse versuchen zwei Semaphoren in unterschiedlicher Reihenfolge zu blockieren.

\subsection{Given the following producer/consumer example program, how could you guard it with semaphores?}
(Annahme: genau ein Producer und ein Consumer.)

\begin{minipage}{.5\textwidth}
    \begin{lstlisting}
void producer(void)
{
    int item;
    while(1) {
        produce_item(&item);
        if (count == N) suspend();
        insert_item(item);
        count = count + 1;
        if (count == 1) wake(consumer)
    }
}
    \end{lstlisting}
    \end{minipage}
    \begin{minipage}{.5\textwidth}
    \begin{lstlisting}
void consumer(void)
{
    int item;
    while(1) {
        if (count == 0) suspend();
        remove_item(&item);
        count = count - 1;
        if (count == N-1) wake(producer);
        consume_item(item);
    }
}
    \end{lstlisting}
    \end{minipage}

Zwei Semaphoren einfügen: einen für den count und einen für N-count (freien Plätze).
Der Producer wird dann die count Semaphore erhöhen und auf die free Semaphore warten, der Consumer wird genau das Gegenteil tun.

\subsection{Describe the Role of the Dispatcher Objects.}
Es sind Objekte (im Windows Kern) auf die man warten kann.
Ein Dispatcher Objekt kann zwei verschiedene Zustände haben: signalisiert oder nicht signalisiert.

\subsection{What does it mean for a Dispatcher Object to be Signaled, or Non-Signaled?}
Bei nicht Signalisierten Objekten: wartende Threads werden blockiert.\\
\unsure Bei signalisierten Objekten: ein wartender Thread kann weiterarbeiten.

\subsubsection{Describe for one type of dispatcher object object what can cause it to change these states.}
Bei Semaphoren: Status ist abhängig vom Wert der Semaphore.\\
Bei File Objekten: wir können mit einem wait auf das File Object explizit darauf warten, dass IO Operationen auf die Datei fertig werden.\\
Bei Prozess Objekten: wir können darauf warten, dass der Prozess terminiert.\\
Bei Timer Objekten: in periodischen Abständen signalisiert.

\subsection{Name two functions of the Windows or UNIX API related to process synchronization and explain their purpose.}
diverse Möglichkeiten: sem\_wait + sem\_signal (Semaphore Operationen), wait (warten auf Prozess), WaitForSingleObject (warten auf Dispatcher Objekt in Windows) pthread\_join (warten auf Thread), \dots

\subsection{What is the IRQL?}
Interrupt Request Level - beschreibt die Wichtigkeit des aktuell von der CPU behandelten Interrupts.
Interrupts unter diesem Level werden später behandelt, Interrupts mit höherem Level sofort.
(Ist somit ein Zustand der CPU.)
Liegt zwischen 0 und 32.

\subsection{Describe what tasks the operating system handles at various ranges of IRQLs.}
0: normaler Zustand\\
1: (spezielle) APCs (Asyncronous Procedure Call)\\
2: Speicherverwaltung, Scheduling, \dots\\
3+: diverse Geräte\\
in den oberen: PowerFail, Bluescreen

\subsection{What is a Trap?}
Der Mechanismus der bei einer einkommenden Unterbrechung den aktuellen Thread Zustand sichert.
(Wir fangen den Thread ein.)

\subsection{What kinds of event can cause a Trap?}
Alle Interrupts, sysenter / syscall, Exceptions, Page Faults

\subsection{What happens if during interrupt processing an interrupt of lower precedence arrives? What if the precedence is higher?}
Wenn die Priorität kleiner ist, dann wird der einkommende Interrupt von der CPU ignoriert und werden behandelt, sobald die IRQL wieder sinkt.\\
Wenn die Priorität höher ist, wird der aktuelle Zustand mittels einer Trap gesichert und der einkommende Interrupt wird behandelt.

\subsection{What are the roles of the Defered and Asynchronous Procedure Calls? (DPC and APC)}
Beide werden verwendet, falls die Interrupt Service Routine lange dauert.
Dann kann diese in eine DPC Objekt verpackt werden und später ausgeführt werden.
(Wenn der IRQL wieder gesunken ist.)
Damit werden andere Interrupts nicht unnötig blockiert uns eventuelle Verluste dieser Interrupts werden vermieden.\\
DPC sind dabei Prozessunabhängig, können also in jedem Thread ausgeführt werden, während APC zu einem Prozess zugehörig sind.
(Entsprechend sind die Speicheradressen die vom Prozess und identisch zum Zeitpunkt des Interrupts.)

\subsection{A Thread on Windows executes the Function ReadFile(). Describe the flow of activity from the application to the device and back.}
\textit{Diese Frage muss in der Klausur nicht so ausführlich beantwortet werden.}

Die ReadFile Funktion ruft die NTReadFile Funktion in Ntdll.dll auf, diese wird einen Systemaufruf zur Kernel Funktion NTReadFile ausführen.
(Dieser Systemaufruf wird wiederum durch einen Softwareinterrupt realisiert, welche eine ISR aufruft.)
Diese wird in der Handle Tabelle die entsprechende Datei zum Handle raussuchen.
Anschließend wird der Treiber für das Gerät auf dem sich die Datei befindet geladen und ausgeführt.
Der Interrupt ist nun fertig und es wird in den User Mode gewechselt.
Der aktive Thread wird erst einmal warten.
Sobald die Festplatte fertig ist mit dem Lesen der Datei, wird sie ein Interrupt schicken, welche einen DPC erzeugt, der die Daten aus der Platte in den Kernelspeicher kopieren wird.
Anschließend erzeugt dieser DPC einen APC (im User Mode), welcher die Daten aus dem Kernelspeicher in den Prozessspeicher kopieren wird.
Anschließend wird der APC den Prozess signalisieren, dass er weiterarbeiten kann, was dann wiederum die ReadFile Funktion beenden wird.

\subsection{What is a pipe? Why is it needed?}
Prozesse sind isoliert, müssen aber eventuell miteinander kommunizieren.
Die Pipe ist ein Verbindungsstück zwischen zwei Prozessen.

\subsection{What is the difference between a pipe and a socket?}
Es sind beides Konstrukte, welche zu Prozessen gehören.\\
In der Pipe ist die Kommunikation nur in eine Richtung.\\
Ein Socket realisiert eine Server-Client Struktur.
(Es gibt einen Server-Prozess welcher einen Socket öffnet und Client-Prozesse die sich mit diesem Socket verbinden können.)
Mit einem Socket können wir in mehrere Richtungen kommunizieren.

\subsection{Pipes and sockets are tied to the lifetime of a process. What other UNIX construct could you use that survives its creating process, and where is it persisted?}
Named Pipes (FIFO): verhält sich ähnlich zu Dateien, ist aber als Puffer implementiert.

\subsection{Write in Pseudocode (Windows or UNIX semantics) a program that launches two processes and connects them through a pipe.}
\missing

\subsection{Name two functions of the Windows or UNIX API related to inter-process communication, and explain their purpose.}
pipe: Erstellen einer Pipe, mkfifo: Erstellen einer Named Pipe, \dots
