\section{Checkpoint 2}

% align with slide numbers
\addtocounter{subsection}{1}

\subsection{What is Preemption?}
In einem System in dem Prozesse in Zeitscheiben unterteilt werden, ist dies die faire Aufteilung dieser Zeitscheiben auf die Prozesse, durch das Unterbrechen des aktuellen Prozesses vom Betriebssystem.
Jeder Prozess erhält ungefähr gleich viel Rechenzeit.

\subsection{What new challanges did Preemption introduce when compared to cooperative multiprogramming?}
Programme können an beliebigen Stellen unterbrochen werden.
Dies wird zu einem Problem, falls eine Resource zwischen verschiedenen Programmen geteielt wird und beide Programme darauf zugreifen.

\subsection{How is Preemption implemented in an operating system kernel?}
Meist durch einen Timer Interrupt. (Eine Zeitscheibe endet, wenn die Clock ein Signal gibt.)
Der Scheduler wählt ein neues Programm aus, welches Rechenzeit erhalten soll und der Dispatcher tauscht den aktiven Kontext aus, damit das neue Programm rechnen kann.

\subsection{Compare Concurrency and Parallelism.}
Concurrency: keine echte Gleichzeitigkeit, sondern eher ein konstantes abwechseln (immer nur aktiver Thread)\\
Parallelism: zwei Kerne, die echt Gleichzeitig rechnen

\subsection{What is a Critical Section?}
Ein Codesegment, in dem auf eine geteilte Resource zugegriffen wird.

\subsection{What is the value of shared, and why?}
In einer Nebenläufigen Ausführung von func\_a und func\_b:

\lstset{language=C}
\begin{minipage}{.5\textwidth}
\begin{lstlisting}
int shared = 0;
void func_a(void)
{
    shared++;
}
\end{lstlisting}
\end{minipage}
\begin{minipage}{.5\textwidth}
\begin{lstlisting}

void func_b(void)
{
    shared--;
}
\end{lstlisting}
\end{minipage}

Antwort: -1, 0 oder 1; je nachdem wie die Unterbrechung der Programme erfolgen, da das keine atomaren Anweisungen sind und wir uns somit temporäre Werte in Registern merken müssen, welche durch die Unterbrechung invalide werden.
