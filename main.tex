\documentclass[a4paper,12pt,oneside]{article}
%for changing amount fo columns in Stichwortverzeichnis
\usepackage{imakeidx}
\usepackage[indentunit=10pt,justific=raggedright]{idxlayout}

\usepackage[utf8]{inputenc}
\usepackage[ngerman]{babel} % this is needed for umlauts
\usepackage[T1]{fontenc}    % this is needed for correct output of umlauts in pdf
\usepackage{graphicx}
\usepackage{fullpage}
\usepackage{amsmath}
\usepackage{amsfonts}
\usepackage{subcaption}
\usepackage{amssymb}
\usepackage{amsthm}
\usepackage{tabularx}
\usepackage{tikz}
\usepackage{enumitem}
\usepackage{dirtree}
\usepackage{listings}
\usepackage{fancyvrb}
\usepackage{ulem}
\usepackage{makecell}
\usepackage{multirow}
\usepackage{color}
\usepackage{geometry}
\usepackage{lscape}
\usepackage{caption}
\usepackage{pgfgantt}
\usepackage{longtable}
\usepackage[explicit]{titlesec}
\usetikzlibrary{automata, arrows, positioning, shapes.geometric, patterns, decorations, decorations.pathreplacing}
\usepackage{mdframed}
\usepackage{pdfpages}
\usepackage{dsfont} % \mathds{N} etc

%spacing of TOC
\usepackage{tocbasic}
\DeclareTOCStyleEntry[
  beforeskip=.2em plus 1pt,% default is 1em plus 1pt
  pagenumberformat=\textbf
]{tocline}{section}

\usepackage{multicol} % For breaking the toc

% thicker overline with \myoverline{arg1} instead of \overline{text}
\usepackage[usestackEOL]{stackengine}
\usepackage{scalerel}
\def\myoverline#1{\ThisStyle{%
  \setbox0=\hbox{$\SavedStyle#1$}%
  \stackengine{.7\LMpt}{$\SavedStyle#1$}{\rule{\wd0}{2\LMpt}}{O}{c}{F}{F}{S}%
}}
\definecolor{carrotorange}{rgb}{0.93, 0.57, 0.13}
\definecolor{cadmiumgreen}{rgb}{0.2, 0.8, 0.2}

\usepackage{makeidx} 
\usepackage[hypertexnames=false]{hyperref}


% Theorem Definition Style
\theoremstyle{definition}
\newtheorem{definition}{Definition}
\newtheorem{theorem}{Theorem}

% Framed Environment Definition
\newenvironment{fdefinition}
{\begin{mdframed}[linewidth=0.2mm, innerbottommargin=0.5cm]\begin{definition}}
		{\end{definition}\end{mdframed}}

% Framed Environment Theorem
\newenvironment{ftheorem}
{\begin{mdframed}[linewidth=0.2mm, innerbottommargin=0.5cm]\begin{theorem}}
		{\end{theorem}\end{mdframed}}

% Solo Framed Environment
\newenvironment{cframe}{\begin{mdframed}[linewidth=0.2mm]}{\end{mdframed}}

% Verbatim style
\RecustomVerbatimCommand{\VerbatimInput}{VerbatimInput}%
{fontsize=\footnotesize,
	%
	frame=lines,  % top and bottom rule only
	framesep=2em, % separation between frame and text
}

% Code blocks
\lstdefinestyle{customc}{
	belowcaptionskip=1\baselineskip,
	breaklines=true,
	xleftmargin=\parindent,
	language=C,
	showstringspaces=false,
	basicstyle=\footnotesize\ttfamily,
	keywordstyle=\bfseries\color{green!40!black},
	commentstyle=\itshape\color{purple!40!black},
	identifierstyle=\color{blue},
	stringstyle=\color{orange},
}
\lstset{escapechar=@,style=customc}

% Custom column type for typewrite styled text columns in tabular
\newcolumntype{T}{>{\ttfamily}l}

% Red colored text
\newcommand{\mcolor}[1]{\textcolor{red}{#1}}

% Padding & Co
\geometry{
	left=1.5cm,
	right=1.5cm,
	top=2cm,
	bottom=3cm,
	bindingoffset=5mm
}

% Circled numbers
\newcommand*\circled[1]{\tikz[baseline=(char.base)]{
		\node[shape=circle,draw,inner sep=2pt] (char) {#1};}}

\newcommand{\cindex}[1]{\index{#1}#1}

\makeindex

\begin{document}

\title{Betriebssysteme}
\date{Wintersemester 2019/20}


\renewcommand{\contentsname}{Inhaltsverzeichnis\vspace*{-0.3cm}}

% %Encomment the following to disable links in TOC
%\makeatletter
%\let\Hy@linktoc\Hy@linktoc@none
%\makeatother

\maketitle
%\begin{frame}{}
\begin{multicols}{2}
  \tableofcontents
  \end{multicols}
%\end{frame}
% Vorlesungen
\newpage
\addcontentsline{toc}{part}{Grundlagen Betriebssysteme} %\part{Grundlagen Betriebssysteme}
%\include{./TeX_files/einleitung1}
%\include{./TeX_files/prozessverwaltung2}
%\include{./TeX_files/sync_prozesse3}
%\include{./TeX_files/speicherverwaltung4}
%\include{./TeX_files/dateisysteme5}
\section{Checkpoint 1}

\subsection{align with slide numbers}

\subsection{What is an Operating System?}
Keine klare Definition vorhanden.
Mögliche Antworten:
Es ist eine Schicht zwischen Hard- und Software, welche Programme und Resourcen verwaltet.

\subsection{Why did Operating Systems emerge?}
Anforderungen an Computer haben sich mit der Zeit verändert.
Jobverwaltung und bessere Auslastung wurde immer relevanter.
Betriebsysteme wurden entsprechend zur Verwaltung verschiedener Aufgaben auf einem System.

\subsection{Which of the following statements is false?}
\begin{enumerate}
    \item[a:] Operating System evolution required Hardware changes
    \item[b:] Hardware evolution required Operating System changes
    \item[c:] Hardware and OS evolved independently
    \item[d:] There was strong influence in both directions 
\end{enumerate}

Lösung: Antwort c ist falsch.

\subsection{Describe three critical early inventions in Operating Systems}
mögliche Lösungen sind:
\begin{enumerate}
    \item Time Sharing
    \item (Pipelines)
    \item Job Controll (mehrere Programme direkt hintereinander ausführen)
    \item Multiprogrammierung (andere Programme laufen, während ein Programm auf IO Events wartet)
\end{enumerate}
\todo auf entsprechenden Folien verweisen

\subsection{What is UNIX?}
Eine historische Betriebssystemfamilie, die auch heute noch immer aktuell ist.

\subsection{What is POSIX?}
Eine stark an UNIX angelehnter Quellcodestandard.
Beschreibt wie ein System entwickelt werden muss, damit es POSIX Kompatibel ist.
(Sich wie ein POSIX System verhält.)
Es ist selbst kein Betriebssystem.

\subsection{Discuss wether POSIX is still relevant today.}
Keine falsche Antwort, da Diskussion.
Entscheidend ist die Begründung.

\subsection{Describe the relation between the terms Process, Program, Thread and File}
Ein Prozess ist ein laufendes Programm, ein Programm ist eine ausführbare Datei und ein Thread ist ein Ausführungsstrang eines Prozesses.
Ein Prozess kann mehrere Threads haben, welche alle im gleichen Kontext laufen. (Sie teilen sich Speicher etc.)

\subsection{What is a shell?}
Eine (Eingabe- und Ausgabe-) Umgebung in der eine Eingabe als Befehl interpretiert und entsprechend ausgeführt wird.
(Kommandointerpreter)

\subsection{Describe briefly what happens when you type `ls' into a UNIX shell and press enter}
Wenn Enter gedrückt wird, wird ein Interrupt erzeugt, welcher vom Betriebssystem abgefangen wird und dann als Eingabe an die Shell gesendet wird.
Anschließend wird der Befehl `ls' von der Shell als Programmaufruf interpretiert woraufhin die Shell sich selbst forkt und im Kind das Programm `ls' im PATH sucht und ggf.\ startet.
`ls' nutzt die gleiche Ausgabe wie die Shell und zeigt so alle Dateien.
Der Elternprozess wartet darauf, dass der Kindprozess terminiert.

\subsubsection{What would have happened if you had typed `cd' instead?}
`cd' wird als aufruf eines Builtin interpretiert, weshalb die Shell in das HOME Verzeichnis des Nutzers wechselt.
Es geschieht kein Fork-Exec.

\subsection{In Pseudo-Code, write a program that executes `ls' in a child process and waits in the parent process for the termination of the child process.}

\lstset{language=C}
\begin{lstlisting}
#include <...>

int main(){
    pid_t pid = fork();
    if(pid == -1){
        printf("error in fork\n");
        return -1;
    }else if(pid == 0){
        // CHILD
        char *const args = {"ls", NULL};
        int ret = execvp("ls", args);
        // ret == -1
        printf("error in exec; errno: %d\n", errno);
        return -1;
    }else{
        // PARENT
        wait(pid);
        return 0;
    }
}
\end{lstlisting}

%% TODO Export in externe Datei

\subsubsection{Which system calls would you use?}
wait, fork, exec

\subsection{Briefly describe five tasks of an Operating System}
Resourcenverwaltung (Storage Managment (Haupt- und Plattenspeicher), Memory Managment, Scheduling (CPU Verwaltung, Prozessormanagment), Device Managment (Geräterverwaltung und Treiber, Interruptbehandlung)),
Security- und Usermanagment (auch Prozessschutz voreinander)

\subsection{Briefly describe three design goals of an Operating System}
Portability, Maintability, Security, Performance, Responsive

\subsection{Discuss the reasoning behind the seperation of User- and Kernel Mode. Identify advantages and disadvantages.}
Schutz der Programme voneinander (Isolation), Benutzerverwaltung und Berechtigungssystem

Ein Nachteil: extra Performance durch ständige Wechsel über Syscalls

\subsection{What is an Interrupt?}
Eine Unterbrechung des aktuellen Programmablaufes (meist durch externe Geräte).
Syncrone Interrupts entstehen durch Programmfluss (meist durch Exceptions) und Asyncrone Interrupts entstehen durch Hardwareereignisse.

\subsection{What can cause an Interrupt?}
Ein Interrupt kann zum Beispiel ausgel\"ost werden, wenn ein Prozess auf IO Zugriff warten muss.

\subsection{Describe how the Operating System handles incoming Interrupts.}
CPU hält die Ausführung des laufenden Prozesses an, die CPU sichert den Kontext des aktuellen Prozesses und  behandelt anschließend den Interrupt entsprechend.
(Meist ist der entsprechende Behandlungscode im Treiber)
In der Interruptbehandlung wird dem Gerät auch mitgeteilt, dass es jetzt aufhören kann Interrupts zu senden.
(Meist gleich als erstes)
Anschließend wird das Programm fortgesetzt, indem sein Kontext wiederhergestellt wird.
Während der Interruptbehandlung wird Interruptbehandlung weiterer möglicherweise eintretenden Interrupts deaktiviert.
Entsprechend darf es während der Interruptbehandlung keine Programmierfehler geben.
Interrupts können auch erst später behandelt werden, je nach Priorität.

%\subsection{Briefly describe five tasks of an Operating System.}
%\begin{itemize}
%	\setlength\itemsep{-0.5em}
%	\item Bereitstellen einer Abstraktionsebene zwischen Hardware und Software
%	\item verwalten der Benutzerrechte
%	\item Bereitstellen einer grafischen Oberfl\"ache
%	\item Parallelit\"at und Nebenl\"aufigkeit
%	\item Verwalten der Hardware Resourcen
%\end{itemize}
%
%\subsection{Briefly describe three design goals of an Operating System.}
%\missing
%
%\subsection{Discuss the reasoning behind the separation of User- and Kernel Mode. Identify advantages and disadvantages.}
%Der Grund f\"ur die Trennung in Kernel und User Mode ist die Sicherheit. Der Benutzer soll keinen direkten Zugriff auf den Kernel haben. \\Vorteil: Sicherheit \\Nachteil: Perfomance Overhead 
%
%\subsection{UPS hab doppelt gearbeitet xD}

\subsection{What is the purpose of a System Call?}
System Calls werden verwendet um von dem User Mode aktionen im Kernel Mode zu initialisieren.

\subsection{Describe what happens when at runtime when a program uses the function `getpid'.}
Wir bekomme die PID(werden ben\"otigt um Prozessen von einander zu entscheiden) des aktuellen Prozesses. Funktion ist in der User Mode Bibliothek beschrieben. Funktion wird aufgerufen und schreibt in ein Register die Syscall Nummer. Cpu wechselt in den Kernel Mode und die Routine f\"ur Systemaufrufe laden. Schaut die Syscall Nummer im Register nach und f\"uhrt dann die entsprechende Funktion hier getpid() aus. Systemaufruf ist ein Synchroner Interrupt.\\
Funktion $\rightarrow$ User Mode Bibliothek $\rightarrow$ Syscall, Softwareinterrupt $\rightarrow$ Interraptionsroutine bestimmt Funktion an Hand der Nummer im Register $\rightarrow$ Funktion ausf\"uhren und Ergebniss nach Oben schleifen

\subsection{What are Windows ``Personalities''?}
\begin{enumerate}
	\item[a:] Independent operating modes of the CPU
	\item[b:] Distinguished Engineers at the Microsoft Campus in Redmond
	\item[c:] Seperate classes of applications, and their corresponding subsystems
	\item[d:] Groups of Users in the System with different Privileges
\end{enumerate}
Antwort C

\subsection{Describe the anatomy of a Windows Subsystem.}
Programme benutzen Subsystems an der Stelle von APIs. Die Subsysteme sind dokumentieren und greifen auf undokumentierte Windows System Service Calls zu.

\subsection{Compare the three types of Subsystem Service Calls.}
\begin{itemize}
	\setlength\itemsep{-0.5em}
	\item vollst\"andig im User Mode implementiert $\rightarrow$ Geometriefkt. PtInRect() IsRectEmpty()
	\item es werden ein oder mehrere Systemcalls f\"ur die Ausf\"uhrung ben\"otigt $\rightarrow$ ReadFile() WriteFile(); erstellt ind Subsystem Library
	\item ben\"otigt die Umgebung des Subsystem Process $\rightarrow$ CreateFile(), durch IPC 
\end{itemize}

\subsection{Name three Operating System components usually located in User- and Kerner Mode (three each)}
\begin{itemize}
	\setlength\itemsep{-0.5em}
	\item User Mode
	\begin{itemize}
		\item Commandointerpreter (Shell)
		\item Subsysteme
		\item Services, Dienste
	\end{itemize}
	\item Kernel Mode
	\begin{itemize}
		\item Grafikengine
		\item Hardware Abstraction, Treiber, ISR
		\item Kernel
		\item Memorymanagement
		\item Management\dots
	\end{itemize}
\end{itemize}

\subsection{What is the role of a System Thread in Windows? Name two examples.}
Thread ist ein Handlungsstrang eines Prozesses im User Mode. \\
Ein Systemthread ist eine nebenläufige Aktivität im Kernel Mode, welche eine spezielle Aufgabe erfüllt.
Ein Systemthread wird durch das System nicht als Prozess abgebildet.
Ein Systemthread hat priviligierten Zugriff auf das System.

Beispiele: Zero Page Thread (Nullt Speicher), eigene Threads von Treibern, Balance Set Manager (greift bei Speicherknappheit ein und verwaltet Prozessorspeicher), generell alle nebenläufigen Aufgaben im Kernel

\subsection{What is a Servive / Daemon?}
Hintergrundprozess im System, Kind von init, hat keine Ein- oder Ausgabe, vollführt Systemrelevante Aufgaben, üblicherweise ist er nicht-priviligiert

\subsection{Compare the concepts ``Microkernel'' and ``Monolithic Kernel''.}
Microkern: so viel Kernel Funktionalität wie möglich in den User Mode verlegen.
Der Kernel enthält nur noch die wirklich relevanten Dinge. (Scheduling, Speicherverwaltung, Dispatch, Interprozesskommunikation)
Meist aus Sicherheitsgründen.
Entsprechend sind alle Treiber und die Kernel-Objektverwaltung im User Mode und werden isoliert.
Zusammenfassung: kleiner Kern mit wenig Funktionen, vielen Diensten

Monolithic Kernel: mächtiger Kern mit vielen Funktionen im Kernel Mode, wenig Diensten

\subsubsection{Which one would you use to describe Windows and which one for Linux?}
Linux: Monolith
Windows: (hybrider Kern,) hauptsächlich Microkern

\subsection{The Windows Kernel is Object Oriented. What is a Handle and what is its role?}
Ein Handle ist ein Verweis auf ein existierendes Kernel Obejekt.
Ein Handle gilt nur für jeweils einen Prozess. (Jeder Prozess hat eine eigene Handle Tabelle.)
Ein Handle kann genutzt werden, um Systemfunktionen Kernel Objekten zu übergeben ohne das der Kernel diese Objekte ``preisgeben'' muss.

\subsection{Can you share handles between separate Processes?}
Nicht direkt. Aber ich kann in die Handle Tabelle eines anderen Prozesses eine weiteres Handle hinzufügen, welches auf das gleiche Kernel Objekt verweist.
(Mit duplicateHandle oder so.)

\subsection{Describe four types of Windows Kernel Objects.}
File Object (einer geöffneten Datei), Port Object (kann genutzt werden um Nachrichten zwischen Prozessen zu verschicken), Thread Object, Process Object, Symbolic Link, Object directory (kann andere Objekte speichern um Hierarchie zu realisieren), Event Object, Semaphore Object, \dots
(siehe Unit 1, Vorlesung 3, Folie 30ff.)

\subsection{What is the Basic Input/Output System (BIOS)?}
Eine Software welches direkt auf der Hardware (der ROM (Read Only Memory)) liegt, welches das aktuell laufende Motherboard booten soll.


\renewcommand{\indexname}{Stichwortverzeichnis} 

% Stichwortverzeichnis soll im Inhaltsverzeichnis auftauchen 
\addcontentsline{toc}{section}{Stichwortverzeichnis} 

% Stichwortverzeichnis endgueltig anzeigen 
\idxlayout{columns=3}
\printindex

\end{document}